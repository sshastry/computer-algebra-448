
\section{Rewriting Systems}

\begin{ap} Many questions we would like to ask about finitely presented groups
    are known to not have general algorithmic solutions. In other words, it has
    been shown that it is not possible to write down an algorithm which solves
    the problem in question.

    The most basic of these unsolvability results pertains to the word problem.
    Let $M = \Mon{X|\mathcal{R}}$ be a finitely presented monoid. The word
    problem for $M$ is to decide, given $U,V\in X^*$, whether $U$ and $V$
    define the same element of $M$. Put differently, we must decide whether $U
    \sim V$ where $\sim$ is the congruence on $X^*$ generated by $\mathcal{R}$.
    It has been shown that there is no general algorithm which will solve the
    word problem for all monoids, and moreover it has been shown there are
    specific monoids for which no solution to the word problem exists. Likewise
    for groups. See Chapter 12 of Rotman, ``The Theory of Groups'' (2e) for a
    proof of this fact.

    Now, it may not be possible to \emph{decide} whether two words $U$ and $V$
    define the same element of $M$. However, if $U$ and $V$ do represent the
    same element, then it is possible to \emph{verify} this fact because the
    definition of the congruence $\sim$ generated by $\mathcal{R}$ is explicit
    enough for us to simply write down all the words in the equivalence class
    containing $U$. This is the content of the next proposition.
\end{ap}

\begin{prop} Let $A$ and $B$ be words in $X^*$. Then $A\sim B$ iff there is a
    sequence of words \[A = A_0, A_1, \dots, A_t = B\] such that for $0 \le i <
    t$ the words $A_i$ and $A_{i+1}$ have the form $CPD$ and $CQD$ respectively
    with $(P,Q)$ or $(Q,P)$ in $\mathcal{R}$.
\end{prop}
\begin{proof} Write $A\equiv B$ if there is such a sequence. Then one checks
    that $\equiv$ is an equivalence relation. If $(P,Q)\in \mathcal{R}$ then
    $P\sim Q$ so that $CPD \sim CQD$ for all $C,D\in X^*$. Thus $A\equiv B
    \Rightarrow A \sim B$, or in other words, we have $\equiv\; \subset\;
    \sim$.

    Next, let us suppose that $A\equiv B$ and let $A_0,\dots, A_t$ be the
    corresponding sequence. If $U$ is a word then the sequence $A_0U, A_1U,
    \dots A_tU$ shows that $AU \equiv BU$. Likewise $UA \equiv UB$. Thus
    $\equiv$ is a congruence. If $(P,Q)\in \mathcal{R}$ then by definition
    $P\equiv Q$. Therefore we have $\sim\; \subset \;\equiv$ and thus $\equiv\;
    =\; \sim$, as required.
\end{proof}

\begin{ap} We obtain the sought verification algorithm as follows. To list all
    words $W$ such that $U\sim W$ then write down all words that can be reached
    using $\mathcal{R}$ by a sequence of length 1, then those which can be
    reached by a sequence of length 2, and so on. Eventually, every word in
    $[U]$ will appear in the list. Therefore, if $U\sim V$ then $V$ will be
    listed and this procedure terminates.

    The unsolvability of the word problem means that although we can list the
    elements of $[U]$, we cannot list the elements of $X^*-[U]$. If it were
    possible to list the elements of the latter, then given $V$, we could run
    both procedures and we would know in a finite amount of time whether $V$
    was in $X^*-[U]$ or in $[U]$. In other words, we could decide whether or
    not $U \sim V.$
\end{ap}

\begin{defn} Let $S$ be a set. A \emph{linear ordering} of $S$ is a transitive
    relation $<$ on $S$ such that for any $s,t\in S$, exactly one of the
    following holds: $s < t, s = t, t < s$. Write $s \le t$ if $s < t$ or $s =
    t$.

    The ordering is a \emph{well-ordering} provided that there does not exist
    an infinite decreasing sequence $s_1 > s_2 > \cdots $ of elements of $S$.
\end{defn}

\begin{prop} If $<$ is a well-ordering on $S$ then every nonempty subset of $S$
    has a least element.
\end{prop}
\begin{proof} Exercise.
\end{proof}

\begin{ap} If $<$ is a linear ordering of a set $S$ and $n$ is a positive
    integer then we may impose the lexicographic ordering on $S^n$ by defining
    $(s_1,\dots,s_n) < (t_1,\dots,t_n)$ iff there is an $i$ with $1 \le i \le
    n$ such that $s_j = t_j$ for $1\le j < i$ and $s_i < t_i$. We warn the
    reader that we use the same symbol $<$ for the ordering on both $S$ and on
    $S^n$. This ordering on $S^n$ is also sometimes called the left-to-right
    lexicographic ordering.
\end{ap}

\begin{prop} Given a linear ordering $<$ on $S$, the corresponding
    lexicographic orderings on $S^n$ are linear orderings, which are moreover
    well-orderings if $<$ is a well-ordering on $S$.
\end{prop}
\begin{proof} Exercise.
\end{proof}

\begin{defn} Given a linear ordering $<$ on a set $X$ we define the
    \emph{lexicographic ordering} on $X^*$ as follows. Let $U = u_1\cdots u_m$
    and $V = v_1 \cdots v_n$ be in $X^*$ with each of the $u_i, v_j \in X$. We
    say that $U< V$ provided that one of the following holds:

    (i) $m < n$ and $u_i = v_i$ for all $1 \le i \le m$.

    (ii) There is an $i$ with $1 \le i \le \min(m,n)$ such that $u_t = v_t$ for all $1 \le t \le i$ and $u_i < v_i$ in $X$.

    Equivalently, $U < V$ iff $U$ is a proper prefix of $V$ or some prefix of
    $U$ is less than the prefix of $V$ of the same length $i$ in the
    lexicographic ordering on $X^i$.
\end{defn}

\begin{prop} If $<$ is a linear ordering on $X$ then the lexicographic ordering
    is a linear ordering on $X^*$.
\end{prop}
\begin{proof} Exercise.
\end{proof}

\begin{ap} If $|X|>1$ then the lexicographic ordering on $X^*$ is not a
    well-ordering, even if $X$ is well-ordered. For instance, if $a,b\in X$ and
    $a<b$ then $ab > a^2b > a^3b > \cdots$ is a strictly decreasing sequence in
    $X^*$.
\end{ap}

\begin{ap} For $w \in X^*$ write $w(i)$ for the prefix of $i$ of length $i$ and
    write $|w|$ for the length of $w$. We define the \emph{length-lexicographic
    ordering} as follows. Let $U,V\in X^*$. Then $U<V$ provided that either
    $|U| < |V|$ or that $|U|=|V|$ and $U < V$ in the lexicographic ordering on
    $X^m$ where $m = |U| = |V|$.
\end{ap}

\begin{prop} The length-lexicographic ordering is a linear ordering on $X^*$
    and if $<$ is a well-ordering of $X$ then $X^*$ is also well-ordered.
\end{prop}
\begin{proof} Let us show that a $<$ well-ordering on $X$ induces a well
    ordering on $X^*$. Let $U_1 > U_2 > \cdots $ be a strictly decreasing
    sequence of words in $X^*$. Since $|U_i| \ge |U_{i+1}|$ for all $i$, from
    some point on all of the $U_i$ have the same length $m$. But then we are
    reduced to the lexicographic ordering on $X^m$ which is a well-ordering
    since $<$ on $X$ is a well-ordering. Thus the sequence terminates.

    The rest of the assertions of the proposition are left to the reader.
\end{proof}

\begin{defn} An ordering $<$ on $X^*$ is \emph{translation invariant} provided
    that \[U < V \Rightarrow AUB < AVB \text{ for all } A,B\in X^*.\]
    Lexicographic orderings are not translation invariant: $a < b \Rightarrow a
    < a^2$ but $ab > a^2b$.

    We say that $<$ is consistent with length provided that \[U < V \Rightarrow
    |U| \le |V|.\] The length-lexicographic orderings are clearly consistent
    with length.
\end{defn}

\begin{prop} The length-lexicographic ordering of $X^*$ is translation invariant.
\end{prop}
\begin{proof} Given words $U < V$ we must show that $Ux < Vx$ and $xU < xV$ for
    all $x\in X$. If $|U| < |V|$ then $|Ux|=|xU| < |Vx| = |xV|$ and thus we
    assume without loss that $m=n$. Suppose that $U$ and $V$ differ first in
    the $i$th term. Then $Ux$ and $Vx$ also differ first in their $i$th terms,
    which coincide with the $i$th terms of $U$ and $V$. Therefore $Ux < Vx$.
    Likewise $xU < xV$.
\end{proof}

\begin{defn} A \emph{reduction ordering} is a translation invariant
    well-ordering.
\end{defn}

\begin{prop} In a reduction ordering on $X^*$ the empty word comes first.
\end{prop}
\begin{proof} If $U < \varepsilon$ for some word $U$ then by translation
    invariance we have $U^2 < \varepsilon U = U$, $U^3 < \varepsilon U^2 < U$,
    and so on. Thus we obtain the infinite strictly descending sequence
    \[\varepsilon > U > U^2 > U^3 > \cdots\] contrary to the well-ordering
    assumption. \end{proof}

\begin{ap} Given a finitely presented monoid $M$ we seek a solution to the word
    problem on $M$ in particular, even though we know that in the word problem
    is unsolvable in general. Suppose that $M = X^*/\sim$ is given as a
    quotient of a free monoid. One way to attack the word problem is to define
    for each element $u\in M$ a word $U\in X^*$, which is in some sense the
    ``simplest'' way of defining $u$. The reduction orderings defined above
    give us a way to make precise the notion that one element is simpler than
    another.
\end{ap}

\begin{defn} Fix a reduction ordering $<$ on $X^*$. For $u\in M$ the set of
    words defining $u$ is nonempty and therefore has a smallest element $U$. We
    define $U$ to be the \emph{canonical form} for $u$ relative to $<$. For a
    word $V$, write $\overline{V}$ for the canonical form of the $\sim$-class
    containing $V$. We remark that the assignment $u\mapsto U$ taking a word to
    its canonical form is a section of the natural projection $X^* \rightarrow
    M$.
\end{defn}

\begin{rem} This definition of canonical class is not constructive and given a
    finitely presented monoid and a reduction ordering, we have at present no
    way of computing a canonical forms. Put differently, we have no algorithm
    which takes $V\in X^*$ as input and produces $\overline{V}\in X^*$ as
    output.

    If we could compute canonical forms, then we would have a solution to the
    word problem because $U\sim V$ iff $\overline{U} = \overline{V}$. Thus, the
    unsolvability of the word problem implies that there are finitely presented
    monoids for which no algorithm $V\mapsto \overline{V}$ exists.
\end{rem}

\begin{prop} If $U$ is the canonical form for an element of $M$ then each
    subword of $U$ is the canonical form for some element of $M$.
\end{prop}
\begin{proof} Let $V$ be a subword of $U$ with $U = AVB$. If $V$ is not a
    canonical form then there is a $W$ such that $V > W$ and $V\sim W$. Then
    $AVB \sim AWB$ and since $<$ is translation invariant, $AVB > AWB$. This
    implies that $U$ is not the least element of its $\sim$-class, a
    contradiction.
\end{proof}

\begin{ap} Let $(P,Q)$ be an element of a generating set $\mathcal{R}$ for
    $\sim$. If we replace $(P,Q)$ by $(Q,P)$ then $\sim$ is unchanged, so we
    may assume without loss that $P>Q$. We call such a $(P,Q)\in \mathcal{R}$
    with $P > Q$ a \emph{rewriting rule} with respect to $<$. If every element
    of $\mathcal{R}$ is a rewriting rule, then $\mathcal{R}$ is called a
    \emph{rewriting system} with respect to $<$.

    Let $\mathcal{R}$ be a rewriting system. Let $\mathcal{P}$ be the set of
    left sides of elements of $\mathcal{R}$, let $\mathcal{N}$ be the ideal of
    $X^*$ generated by $\mathcal{P}$, and put $\mathcal{C} := X^*-\mathcal{N}$.
    For each $U\in \mathcal{N}$ there are $A,B,P,Q$ such that $U=APB$ and
    $(P,Q)\in \mathcal{R}$. Let $V=AQB$. Then $P\sim Q$ so that $U\sim V$.
    Moreover, we have $P>Q$ by definition, so that $U > V$. If $V\in
    \mathcal{N}$ we repeat this procedure to obtain a word $W$ such that $U\sim
    V\sim W$ and $U > V > W.$ Since $<$ is a well-ordering, this process
    terminates and we obtain $C\in \mathcal{C}$ with $U\sim C$ and $U \ge C$.

    The procedure just described wherein we replace subwords which are left
    sides of rewriting rules with the corresponding right sides is called
    \emph{rewriting.} We call the elements of $\mathcal{C}$ \emph{irreducible}
    or \emph{reduced} with respect to $\mathcal{R}$ since no rewriting can be
    performed on them. The canonical form of every $\sim$-class is in
    $\mathcal{C}$.

    Let $(P,Q)\in \mathcal{R}$. We write $P\rightarrow Q$ for $(P,Q)$ and more
    generally write $U \rightarrow V$ or $U\xrightarrow{\mathcal{R}} V$ if
    \emph{$V$ is derived from $U$ in one step using $\mathcal{R}$} by which we
    mean: there are words $A,B,P,Q$ such that $P\rightarrow Q \in \mathcal{R}$,
    $U=APB,$ and $V = AQB.$ Let us give the pseudocode for this procedure.

\begin{algorithm}
\caption{Rewriting of words wrt a rewriting system}
\begin{algorithmic}[1]
    \Procedure{Rewrite}{$X, \mathcal{R}, U$}
    \State Input:
    \State $X$ = finite set;
    \State $\mathcal{R}$ = finite rewriting system on $X^*$ wrt a reduction ordering;
    \State $U$ = word in $X^*$.
    \State Output:
    \State $V$ = word in $X^*$ irreducible wrt $\mathcal{R}$
    \State \;\;\;\;\;\; and definining the same element of \Mon{X|\mathcal{R}} as $U$.
    \State
    \State Let $\mathcal{P}$ be the set of left sides in $\mathcal{R}$;
    \State V := U;
    \While{$V$ contains a subword in $\mathcal{P}$;}
    \State Let $V=APB$ with $P\rightarrow Q$ in $\mathcal{R}$;
    \State $V := AQB$;
    \EndWhile

    \EndProcedure
\end{algorithmic}
\end{algorithm}

    It remains to be shown that the output of the above procedure depends only
    on the input $U$ and not on the choices made in each execution of the body
    of the loop. This will be accomplished in \ref{welldefined}, below.
\end{ap}

\begin{rems} (1) Let us remark on the syntax used in the above pseudocode. The
    expression ``Let $V=APB$'' means given $V$ we write it in the stated form.
    The expression $V := AQB$ means we define $V$ to have a new value. It is
    this new value which is tested in the clause of the next iteration of the
    loop and which is passed in to the body of the loop should the test
    succeed.

    (2) We note for emphasis that it is only the fact that there is a reduction
    ordering in the background that ensures that this procedure always
    terminates in finitely many steps.
\end{rems}

\begin{defn} Let $\mathcal{R}$ be a rewriting system on $X^*$ associated to the
    reduction ordering $<$. Let $\sim$ be the congruence generated by
    $\mathcal{R}$. Let $U,V \in X^*$. We say that \emph{$V$ is derivable from
    $U$ by means of $\mathcal{R}$} if there is a sequence of words \[U =
    U_0, U_1, \dots, U_t = V\] with $t\ge 0$ such that $U_i
    \xrightarrow{\mathcal{R}} U_{i+1}$ for each $0 \le i < t$, and we shall
    denote this phenomenon by \[U \xLongrightarrow{\mathcal{R}} V. \] The
    relation $\xLongrightarrow{\mathcal{R}}$ is reflexive and transitive on
    $X^*$. Moreover, $U \xLongrightarrow{\mathcal{R}} V$ implies that
    $U\sim V$ and $U \ge V$.
\end{defn}

\begin{prop} If $U \xLongrightarrow{\mathcal{R}} V$ then $UW
    \xLongrightarrow{\mathcal{R}} VW$ and $WU \xLongrightarrow{\mathcal{R}} WV$
    for all $W \in X^*$.
\end{prop}
\begin{proof} Unwind the definitions.
\end{proof}

\begin{prop}\label{wordseq} Let $U,V\in X^*$. Then $U\sim V$ if and only if
    there is a sequence of words $U = U_0, U_1, \dots, U_t = V$ such that for
    each $0 \le i < t$ we have either $U_i \xLongrightarrow{\mathcal{R}}
    U_{i+1}$ or $U_{i+1} \xLongrightarrow{\mathcal{R}} U_i.$
\end{prop}
\begin{proof} Write $U\equiv V$ if there is a sequence $U_0, \dots, U_t$ as in
    the statement of the proposition. Since $U_i \sim U_{i+1}$ we have $U\equiv
    V \Rightarrow U\sim V$, or in other words $\equiv \; \subset \; \sim$. One
    checks that $\equiv$ is an equivalence relation. By the previous
    proposition, $\equiv$ is a congruence on $X^*$. For $(P,Q)\in \mathcal{R}$,
    we have $P\equiv Q$. Thus $\sim \; \subset \; \equiv$.
\end{proof}

\begin{defns} (1) We say that the relation $\xLongrightarrow{\mathcal{R}}$ has
    the \emph{Church-Rosser property} if $U\sim V$ implies that there is a word
    $Q$ such that $U\xLongrightarrow{\mathcal{R}} Q$ and $V
    \xLongrightarrow{\mathcal{R}} Q$.

    (2) We say that $\xLongrightarrow{\mathcal{R}}$ is \emph{confluent} if $W
    \xLongrightarrow{\mathcal{R}} U$ and $W \xLongrightarrow{\mathcal{R}} V$
    implies that there is a word $Q$ such that $U\xLongrightarrow{\mathcal{R}}
    Q$ and $V \xLongrightarrow{\mathcal{R}} Q$.

    (3) We say that $\xLongrightarrow{\mathcal{R}}$ is \emph{locally confluent}
    if $W \xrightarrow{\mathcal{R}} U$ and $W \xrightarrow{\mathcal{R}} V$
    implies that there is a word $Q$ such that $U \xLongrightarrow{\mathcal{R}}
    Q$ and $V \xLongrightarrow{\mathcal{R}} Q$.
\end{defns}

\begin{rem} The preceeding definitions are analgous to properties that we might
    ask of open sets in a topological space, as follows. Suppose that $U$ and
    $V$ are open subsets of a topological space $X$. Fix $x\in X$ and define an
    equivalence relation on the open subsets of $X$ by $U \sim V$ iff $x\in
    U\cap V$. Next, suppose that we define $U_1 \xLongrightarrow{\mathcal{R}}
    U_2$ to mean $U_2 \subset U_1$ and $x\in U_2$. Thus $U_1
    \xLongrightarrow{\mathcal{R}} U_2$ implies $U_1 \sim U_2$.

    The Church-Rosser property then is analogous to the basic fact that $U\sim
    V \Rightarrow $ there exists an open set $Q$ contained in both $U$ and $V$
    which contains $x$. Of course, $Q = U\cap V$ will do.

    % The properties of confluence and local confluence likewise have analogues
    % for topological spaces, although admittedly the conditions may not seem
    % so useful from the point of view of topology.

    Loosely speaking, all of the conditions, Church-Rosser, confluence, and
    local confluence are variations on the theme of  ``an intersection of open
    sets is open'' in the context of a neighborhood basis of a point.
\end{rem}

\begin{prop}\label{CR1} If the Church-Rosser property holds then every
    $\sim$-class contains a unique element of $\mathcal{C}$, namely the
    canonical form for said class.
\end{prop}
\begin{proof} As we have seen earlier, the canonical form for each $\sim$-class
    is in $\mathcal{C}$. Let $U,V\in \mathcal{C}$ which lie in the same
    $\sim$-class. The Church-Rosser property tells us that there is a word $Q$
    such that $U\xLongrightarrow{\mathcal{R}} Q$ and $V
    \xLongrightarrow{\mathcal{R}} Q$. Since no nontrivial rewrites can be
    performed on the elements of $\mathcal{C}$, we have $U = V = Q.$
\end{proof}

\begin{thm} \label{CR2} Suppose given a rewriting system $\mathcal{R}$ relative
    to a reduction ordering $<$. The Church-Rosser property, confluence, and
    local confluence are equivalent notions for $\mathcal{R}$.
\end{thm}
\begin{proof} Show that Church-Rosser implies confluence. Given that
    $W\xLongrightarrow{\mathcal{R}} U$ and $W \xLongrightarrow{\mathcal{R}} V$
    we have that $W\sim U$ and $W\sim V$ and therefore $U\sim V$. The
    Church-Rosser property guarantees us the existence of a word $Q$ such that
    $U\xLongrightarrow{\mathcal{R}} Q$ and $V \xLongrightarrow{\mathcal{R}} Q.$

    Show that confluence implies local confluence. This is by definition.

    Show that local confluence implies confluence. Confluence fails for a word
    $W$ if there are words $U,V$ with $W \xLongrightarrow{\mathcal{R}} U, W
    \xLongrightarrow{\mathcal{R}} V$ but no word $Q$ with $U
    \xLongrightarrow{\mathcal{R}} Q$ and $V \xLongrightarrow{\mathcal{R}} Q.$
    Let $\mathcal{W}$ be the set of words at which confluence fails. Suppose
    for contradiction that $\mathcal{W}$ is nonempty. Since $<$ is a
    well-ordering, we can define $W_0 := \min \mathcal{W}$. If $W_0
    \xLongrightarrow{\mathcal{R}} U$ and $W_0 \xLongrightarrow{\mathcal{R}} V$
    then we seek a word $Q$ with $U \xLongrightarrow{\mathcal{R}} Q$ and $V
    \xLongrightarrow{\mathcal{R}} Q$. If $U=W_0$ then we take $Q=V$. If $V=W_0$
    then we take $Q=U$. Therefore, without loss, we may assume that $U\neq W_0$
    and $V\neq W_0$.

    There are words $A,B$ such that $W_0 \xrightarrow{\mathcal{R}} A, W_0
    \xrightarrow{\mathcal{R}} B$ such that $A \xLongrightarrow{\mathcal{R}} U,
    B \xLongrightarrow{\mathcal{R}} V$. Local confluence gives us a word $C$
    with $A \xLongrightarrow{\mathcal{R}} C$ and $B
    \xLongrightarrow{\mathcal{R}} C$. Since $A < W_0$, $A\notin \mathcal{W}$.
    Therefore there exists $D$ with $U \xLongrightarrow{\mathcal{R}} D, C
    \xLongrightarrow{\mathcal{R}} D$. Therefore $B
    \xLongrightarrow{\mathcal{R}} D$. Now, $B < W_0$ implies that there is a
    $Q$ such that $D \xLongrightarrow{\mathcal{R}} Q$ and $V
    \xLongrightarrow{\mathcal{R}} Q$. This implies that $U
    \xLongrightarrow{\mathcal{R}} Q$ and therefore confluence does not fail for
    $W_0$, a contradiction. Thus $\mathcal{W}= \varnothing$, as required.

    Show that confluence implies Church-Rosser. Given $U\sim V$, we must show
    that there is a $Q$ with $U\xLongrightarrow{\mathcal{R}} Q, V
    \xLongrightarrow{\mathcal{R}} Q$. By \ref{wordseq}, we have a sequence
    $U=U_0,U_1,\dots,U_t = V$ such that for all $0 \le i < t$, $U_i
    \xLongrightarrow{\mathcal{R}} U_{i+1}$ or $U_{i+1}
    \xLongrightarrow{\mathcal{R}} U_i$. We now prove the claim by induction on
    $t$. If $t=0$ then $Q = U = V$ will do. If $t=1$ then put $Q := \min(U,V)$.
    Let $t \ge 2$. Then $U_1\sim V$ and the induction hypothesis gives us a
    word $A$ with $U_1 \xLongrightarrow{\mathcal{R}} A, V
    \xLongrightarrow{\mathcal{R}} A.$ If $U_0 \xLongrightarrow{\mathcal{R}}
    U_1,$ then we put $Q:=A$. Suppose that $U_1 \xLongrightarrow{\mathcal{R}}
    U_0$. Confluence gives us $Q$ such that $U_0 \xLongrightarrow{\mathcal{R}}
    Q$ and $A \xLongrightarrow{\mathcal{R}} Q$. Therefore $V
    \xLongrightarrow{\mathcal{R}} Q$, as required.
\end{proof}

\begin{prop}\label{welldefined} If $\mathcal{R}$ is a confluent rewriting
    system on $X^*$, then the result $V$ of \textsc{Rewrite}$(X,\mathcal{R},U)$
    depends only on $\mathcal{R}$ and $U$ and not on the unspecified choices
    made in the execution of the algorithm.
\end{prop}
\begin{proof} By \ref{CR1} and \ref{CR2}, the value $V$ is the canonical form
    for the $\sim$-class containing~$U$.
\end{proof}

\begin{rem} The algorithm \textsc{Rewrite} is only partially defined. This
    proposition says that it is ``defined enough.'' Figuratively speaking, in
    topological terms we might say that there is a topology on the space of
    sample paths which is coarse enough to render irrelevant all of the
    arbitrary choices made in \textsc{Rewrite} and which is such that
    \textsc{Rewrite} is well-defined and continuous with respect to that
    topology.

    We have already had a hint of the precise formulation of these notions in
    the first part of the course on automata theory; the space of sample paths
    of the algorithm in question is akin to a regular language $L \subset A^*$
    where $A$ is an alphabet.
\end{rem}

\begin{defn} A rewriting system $\mathcal{R}$ is said to be \emph{reduced} if
    for each $(P,Q)\in \mathcal{R}$, $Q$ is irreducible with respect to
    $\mathcal{R}$, no word appears as $P$ (i.e.~appears as a left hand side)
    for two different rules, and no left hand side contains another left side
    as a proper subword. Put differently, $\mathcal{R}$ is reduced iff for each
    $(P,Q)\in \mathcal{R}$, both $P$ and $Q$ are irreducible with respect to
    $\mathcal{R}-\{(P,Q)\}.$
\end{defn}

\begin{prop} Let $<$ be a reduction ordering on $X^*$. Every congruence on
    $X^*$ is generated by a unique reduced and confluent rewriting system with
    respect to $<$.
\end{prop}
\begin{proof} Let $\sim$ be a congruence on $X^*$ and let $M$ be $X^*/\sim$.
    Let $\mathcal{C} := \{\overline{U} : U \in X^*\}$ be the set of canonical
    forms for the elements of $M$ and let $\mathcal{N}$ be the ideal
    $X^*-\mathcal{C}$.

    Let $\mathcal{P}$ be the unique minimal generating set for $\mathcal{N}$
    (see \ref{mingenset}). Put $\mathcal{S} := \{(P,\overline{P}) : P \in
    \mathcal{P}\}.$ We now show that $\mathcal{S}$ is the unique reduced
    confluent rewriting system with respect to $<$.

    We have $U \ge \overline{U}$ for any $U$. For $P\in \mathcal{N}, P \neq
    \overline{P}$ so that $P > \overline{P}$. Thus $\mathcal{S}$ is a rewriting
    system with respect to $<$. Let $\equiv$ be the congruence generated by
    $\mathcal{S}$. Since $P \sim \overline{P}$ for all $P\in \mathcal{P}$ we
    have $\equiv \; \subset \; \sim$. Now, any word which is irreducible with
    respect to $\mathcal{S}$ is in $\mathcal{C}$ and therefore
    \textsc{Rewrite}$(X,\mathcal{S},U)$ returns $\overline{U}$. This means that
    $U \equiv \overline{U}$ for all words $U$ and therefore that $\equiv$ and
    $\sim$ coincide. Note that $\mathcal{S}$ is reduced by definition. If $U
    \sim V$ then $U\xLongrightarrow{\mathcal{S}} \overline{U}$ and $V
    \xLongrightarrow{\mathcal{S}} \overline{V}$. Since $\overline{U} =
    \overline{V}$ the Church-Rosser property is satisfied and $\mathcal{S}$ is
    confluent.

    Next, suppose that $\mathcal{T}$ is a reduced and confluent rewriting
    system with respect to $<$ which generates $\sim$. Fix $P \in \mathcal{P}$.
    Using $\mathcal{T}$ to rewrite ${P}$ gives $\overline{P}$. It follows that
    $P$ contains a left side of $\mathcal{T}$ as a subword. On the other hand,
    all proper subwords of $P$ are in $\mathcal{C}$. Therefore $\mathcal{T}$
    contains a rule of the form $(P,Q)$ for some $Q$. The fact that
    $\mathcal{T}$ is reduced forces $Q\in \mathcal{C}$ and therefore $Q =
    \overline{P}$ and we obtain $\mathcal{S} \subseteq \mathcal{T}$. If $(U,V)
    \in \mathcal{T}-\mathcal{S}$ then $U\notin \mathcal{C}$ and it follows that
    $U$ contains a subword $P\in \mathcal{P}$. By definition, $P$ is a left
    side in $\mathcal{S}$, contradicting the fact that $\mathcal{T}$ is
    reduced.
\end{proof}

\begin{defn} Given a reduction ordering $<$ on $X^*$ and $\mathcal{R} \subset
    X^* \times X^*$, let $\sim$ be the congruence generated by $\mathcal{R}$.
    Write $\mathrm{RC}(X,<,\mathcal{R})$ or $\mathrm{RC}(X,<,\sim)$ for the
    reduced and confluent rewriting system which generates $\sim$.

    If $\mathcal{R}$ is assumed to be confluent, the following proposition
    identifies $\mathrm{RC}(X,<,\mathcal{R})$.
\end{defn}

\begin{prop}\label{confluencecharacterization} Let $\mathcal{R}$ be confluent
    on $X^*$ associated to $<$. Put \emph{$$\mathcal{P} := \{ P : (P,Q) \in
        \mathcal{R} \text{ and no left side in } \mathcal{R} \text{ is a proper
        subword of } P \}.$$} Write $\overline{P}$ for the result of rewriting
        $P$ by means of $\mathcal{R}$. Then we have
        \[\mathrm{RC}(X,<,\mathcal{R}) = \{(P,\overline{P}) : P \in
        \mathcal{P}\}.\]
\end{prop}
\begin{proof} Write $\sim$ for the congruence generated by $\mathcal{R}$. The
    confluence of $\mathcal{R}$ guarantees us that any word which is not the
    least element of its $\sim$-class must contain a subword which is a left
    side in $\mathcal{R}$. Therefore $\mathcal{P}$ is the minimal generating
    set for the ideal of noncanonical forms and the result follows.
\end{proof}

\begin{prop} Suppose given a congruence $\sim$ on $X^*$, where as usual, $X$ is
    assumed to be finite. Suppose that the set of $\sim$-classes is finite.
    Then $\mathrm{RC}(X,<,\sim)$ is finite for every reduction ordering $<$ of
    $X^*$.
\end{prop}
\begin{proof} Put $\mathcal{S} := \mathrm{RC}(X,<,\sim)$ and let $(P,Q)\in
    \mathcal{S}$. Recall that we write $w(i)$ for the prefix of length $i$ of a
    given word. Since each $P(i)$ is a canonical form, $P_i$ and $P_j$ are in
    different $\sim$-classes if $i < j < |P|$. It follows that $|P| \le \#
    \{\sim\text{-classes}\}$. Since distinct elements of $\mathcal{S}$ have
    distinct left sides, it follows that $\mathcal{S}$ is finite.
\end{proof}

\begin{defn} If $\mathcal{R}$ is a confluent rewriting system on $X^*$ then we
    say that $\Mon{X|\mathcal{R}}$ is a \emph{confluent presentation} for the
    monoid in question.
\end{defn}

\begin{ap} Given $X$, let $<$ be a reduction ordering on $X^*$ and let
    $\mathcal{R}$ be a finite rewriting system on $X^*$ with respect to $<$. If
    we suppose that $\mathcal{R}$ is confluent then we know how to solve the
    word problem in $M := \Mon{X|\mathcal{R}}$ by using \textsc{Rewrite}. Thus
    if we could test a rewriting system for confluence, then we would have a
    way to say something about the solvability of the word problem for a given
    presentation.

    We say that local confluence fails at a word $W$ if there are $U,V \in X^*$
    such that $W \xrightarrow{\mathcal{R}} U$ and $W \xrightarrow{\mathcal{R}}
    V$ but there is no $Q$ such that $U \xLongrightarrow{\mathcal{R}} Q$ and $V
    \xLongrightarrow{\mathcal{R}} Q.$
\end{ap}

\begin{prop}\label{localfailure} Let $W$ be a word such that local confluence
    fails at $W$ but does not fail at any proper subword of $W$. Then one of
    the following holds:

    (1) $W$ appears as the left side of two distinct elements of $\mathcal{R}$.

    (2) $W$ is a left side in $\mathcal{R}$ which contains another left side as a proper subword.

    (3) $W = ABC$ where $A,B,C$ are nonempty words such that $AB$ and $BC$ are left sides in $\mathcal{R}$.
\end{prop}
\begin{proof} This is Proposition 3.1 on page 58 of Sims.
\end{proof}

\begin{defn} If $W$ is as in the proposition, then we call $W$ an \emph{overlap
    of left sides} in $\mathcal{R}$. If the third condition holds then we say
    that $W$ is a \emph{proper overlap.} Only proper overlaps can arise in a
    reduced rewriting system.
\end{defn}

\begin{ap} Since $\mathcal{R}$ is finite, the set $\mathcal{W}$ of words which
    are overlaps of left sides in $\mathcal{R}$ is also finite. For each $W\in
    \mathcal{W}$, write $\mathcal{U}_W$ for the finite set of words $U$ such
    that $W \xrightarrow{\mathcal{R}} U$ is a derivation consisting of a single
    step. For each $U\in \mathcal{U}_W$ we put $V :=$
    \textsc{Rewrite}$(X,\mathcal{R},U)$. As $U$ varies, if more than one $V$ is
    obtained, then $\mathcal{R}$ is not confluent. The reason is that in this
    case we have found two words which are irreducible with respect to
    $\mathcal{R}$ and define the same element of $M$.

    On the other hand, if only one value of $V$ is seen as $U$ varies in $\mathcal{U}_W$, then local confluence does not fail at $W$.

    Performing this test for all $W\in \mathcal{W}$, we have an algorithm for
    determining whether or not $\mathcal{R}$ is confluent. We now proceed to
    describe this algorithm formally.

\begin{algorithm}
\caption{Testing a rewriting system for confluence}
\begin{algorithmic}[1]
    \Procedure{Confluent}{$X, \mathcal{R}$}
    \State Input:
    \State $X$ = finite set;
    \State $\mathcal{R}$ = finite rewriting system on $X^*$ wrt a reduction ordering;
    \State Output:
    \State True or False according as $\mathcal{R}$ is or is not confluent.
    \State
    \For{$(P,Q)\in \mathcal{R}$}
    \For{$(R,S)\in \mathcal{R}$}
    \For{$B$ a nonempty suffix of $P$}
    \State Let $U$ be the longest prefix common to both $B$ and $R$;
    \State Let $B = UD, R = UE;$
    \If{$D$ or $E$ is the empty word}
    \State $P := AB;$
    \State $V :=$ \textsc{Rewrite}$(X,\mathcal{R},ASD);$
    \State $W :=$ \textsc{Rewrite}$(X,\mathcal{R},QE);$
    \If{$V\neq W$}
    \State return False;
    \EndIf
    \EndIf
    \EndFor
    \EndFor
    \EndFor
    \State return True;
    \EndProcedure
\end{algorithmic}
\end{algorithm}
\end{ap}
\newpage{}

\begin{eg} Take $X := \{x,y,z\}$ and consider the finite rewriting system
    $\mathcal{S} := \{x^2 \rightarrow \varepsilon, yz\rightarrow \varepsilon,
    zy \rightarrow\varepsilon \}.$ We claim that $\mathcal{S}$ is confluent.
    The claim is proved by tracing through the steps of the above algorithm.
\end{eg}

\begin{ap} Given a rewriting system $\mathcal{S}$ and a word $U\in X^{\pm *}$
    there are typically many ways of rewriting $U$ with the objective of
    obtaining an irreducible word. For the sake of writing a computer program
    which actually finds the sought irreducible word, we must fix a way of
    rewriting a given word by using a given rewriting system.

    Let us informally describe a strategy as follows. If we are rewriting the
    word $U$, then we must write $U$ as $U = APB$ and then replace it by $U =
    AQB$ where $(P,Q)\in \mathcal{S}$. How to choose which occurence of $P$ to
    use in $U$? We choose the left side $P$ in $U$ which is maximal with
    respect to $<$ and subject to this condition, as far to the left in $U$ as
    possible. Let us give the code of a refined version of the foregoing
    heuristic ideas.

\begin{algorithm}
\caption{Rewriting from left}
\begin{algorithmic}[1]
    \Procedure{RewriteLeft}{$X,\mathcal{R}, U$}
    \State Input: $X$ = generators, $\mathcal{R}$ = rewriting system, $U$ = a word;
    \State Output: the rewritten form of $U$
    \State $V:= \varepsilon, W := U$;
    \While{$W \neq \varepsilon$}
    \State Let $W = xW_1$ where $x\in X$; $W:= W_1, V:= Vx;$
    \For{$i = 1,\dots,n$}
    \If{$P_i$ is a suffix of $V$}
    \State $V:= RP_i, W := Q_i W, V:= R;$
    \State break
    \EndIf
    \EndFor
    \EndWhile
    \EndProcedure
\end{algorithmic}
\end{algorithm}

We have the following invariants of the algorithm (i.e.~the following
statements are true at each step of the execution of the algorithm):

(1) the word $VW$ is derivable from $U$.

(2) $V$ is the longest prefix of $VW$ known at the time of execution to be
irreducible with respect to the set of rules.

\end{ap}

\begin{eg} Let us trace through the execution of this procedure in an example.
    Let $X = \{a,b\}$ and consider the rewriting system $\mathcal{S} := \{a^2
    \rightarrow \varepsilon, ab^2a \rightarrow bab, b^3 \rightarrow
    \varepsilon, baba \rightarrow ab^2, abab \rightarrow b^2a, b^2ab^2 \rightarrow
    aba\}.$ Here is a ``sample path'' of the execution of \textsc{RewriteLeft}$(X,
    \mathcal{S}, U)$.

    \begin{align*}
        ab\underline{aa}babbaaabbabbbaba \\
        \underline{abba}bbaaabbabbbaba \\
        ba\underline{bbb}aaabbabbbaba \\
        b\underline{aa}aabbabbbaba \\
        b\underline{aa}bbabbbaba \\
        \underline{bbb}abbbaba \\
        a\underline{bbb}aba \\
        \underline{aa}ba \\
        ba
    \end{align*}
\end{eg}

\section{The Knuth-Bendix Algorithm}

\begin{ap} Let $(X,\mathcal{R})$ be a finite presentation of a monoid $M$.
    Given a reduction ordering $<$ on $X^*$, let $\mathcal{T} :=
    \mathrm{RC}(X,<,\mathcal{R})$. Assume that $\mathcal{T}$ is finite. The
    Knuth-Bendix algorithm takes as input $X,\mathcal{R}, <$ and produces
    $\mathcal{T}$. The algorithm makes an essential use of the hypothesis that
    $\mathcal{T}$ is finite; no procedure is given to decide whether or not
    $\mathcal{T}$ is finite. Moreover, the algorithm requires that $<$ be
    effective in the sense that $<$ is given together with an algorithm to
    determine whether or not the statement ``$U < V$'' is true.

    The idea of the algorithm goes as follows. Apply \textsc{Confluent} to the
    current set of rules. If the rules are confluent, we are done. Otherwise
    there exist irreducible words $A$ and $B$ such that $A\sim B$ where $\sim$
    is the congruence generated by $\mathcal{R}$. Changing notation if need be,
    we have $A > B$ and $(A,B)$ is appended to our set of rules.

    We write $\mathcal{S} = \{(P_i,Q_i)\}_{i=1}^n$ for the set of rules that
    have been found so far. When the computation terminates, we will have
    $\mathcal{S} = \mathcal{T}$.

    Let us now give the formal specifications of the algorithm. First we
    require a subroutine \textsc{Update}$(U,V)$ which will, if need be, add a
    new rule so as to ensure that there exists a word derivable from the given
    words $U,V$ by means of the rules in $\mathcal{S}$.

\newpage{}

\begin{algorithm}
\caption{Updating the list of rules}
\begin{algorithmic}[1]
    \Procedure{Update}{$\mathcal{S}, U,V$}
    \State Input: $\mathcal{S} = \{(P_1,Q_1), (P_2,Q_2), \dots,(P_n,Q_n)\}$ a finite rewriting system; $U,V = $ words;
    \State Output: none; the state of $\mathcal{S}$ is modified in place;
    \State $A :=$ \textsc{RewriteLeft}$(U)$;
    \State $B :=$ \textsc{RewriteLeft}$(V)$;
    \If{$A\neq B$}
    \If{$A < B$}
    \State swap $A$ and $B$;
    \EndIf
    \State append $(A,B)$ to $\mathcal{S}$;
    \EndIf
    \EndProcedure
\end{algorithmic}
\end{algorithm}

We need one more subroutine, the procedure \textsc{Overlap}$(\mathcal{S},
i,j)$. A new rule is appended to $\mathcal{S}$ for each failure of local
confluence.

\newpage{}

\begin{algorithm}
\caption{Check the overlaps of $P_i$ and $P_j$ in which $P_i$ is a prefix}
\begin{algorithmic}[1]
    \Procedure{Overlap}{$\mathcal{S}, i,j$}
    \State Input: $\mathcal{S} = \{(P_1,Q_1), (P_2,Q_2), \dots,(P_n,Q_n)\}$; $i,j = $ positive integers $\le |\mathcal{S}|$
    \State Output: none; the state of $\mathcal{S}$ is modified in place;
    \For{$k := 1, \dots, |P_i|$}
    \State Let $P_i = AB$ where $|B| = k$;
    \State Let $U$ be the longest word which is a prefix of both $B$ and $P_j$;
    \State Let $B = UD$ and $P_j = UE;$
    \If{$D = \varepsilon$ or $E = \varepsilon$}
    \State \textsc{Update}$(\mathcal{S}, AQ_jD, Q_iE)$;
    \EndIf
    \EndFor
    \EndProcedure
\end{algorithmic}
\end{algorithm}

\begin{algorithm}
\caption{The Knuth-Bendix Algorithm}
\begin{algorithmic}[1]
    \Procedure{KnuthBendix}{$X, <, \mathcal{R}$}
    \State Input:
    \State $X=$ a finite set, $<$ = reduction ordering on $X^*$, $\mathcal{R} \subset X^*\times X^*$ a finite subset;
    \State Output: $\mathcal{T} = \mathrm{RC}(X, <, \mathcal{R})$ if it is finite
    \State
    \State $\mathcal{S} := \{\};i:=1;$
    \For{$(U,V) \in \mathcal{R}$}
    \State \textsc{Update}$(\mathcal{S},U,V);$
    \EndFor
    \While{$i \le n$}
    \For{$j:=1,\dots,i$}
    \State \textsc{Overlap}$(\mathcal{S}, i,j)$;
    \If{j < i}
    \State \textsc{Overlap}$(\mathcal{S}, j,i)$;
    \EndIf
    \EndFor
    \State $i := i+1;$
    \EndWhile
    \State Let $\mathcal{P} := \{P_i : \text{every proper subword of } P_i \text{ is irreducible wrt } \mathcal{S}\}$;
    \State $\mathcal{T} := \{\}$;
    \For{$P\in \mathcal{P}$}
    \State $Q :=$ \textsc{RewriteLeft}$(X, \mathcal{R}, P)$;
    \State append $(P,Q)$ to $\mathcal{T}$;
    \EndFor
    \EndProcedure
\end{algorithmic}
\end{algorithm}

\newpage{}

\begin{thm} If $\mathrm{RC}(X,<,\mathcal{R})$ is finite then
    \textsc{KnuthBendix}$(X, < \mathcal{R})$ terminates and outputs
    $\mathrm{RC}(X,<,\mathcal{R})$.
\end{thm}
\begin{proof} Tracing the steps of the execution of the algorithm, if the call
    \textsc{Update}$(\mathcal{S},U,V)$ has been made and the two calls to
    \textsc{RewriteLeft} inside of update have completed, then $\mathcal{S}
    \cup \{(U,V)\}$ and $\mathcal{S} \cup \{(A,B)\}$ generate the same
    congruence on $X^*$. It follows that after the execution of the first for
    loop in \textsc{KnuthBendix} (i.e.~at line 11) the sets $\mathcal{R}$ and
    $\mathcal{S}$ generate the same congruence on $X^*$. In particular,
    $\mathcal{S}$ generates $\sim$ at any time during the execution of the
    while loop starting on line 12.

    We observe that $P_i$ is irreducible with respect to $\{(P_j,Q_j) : 1 \le j
    < i\}.$

    Suppose that \textsc{KnuthBendix} does not terminate in finitely many
    steps. This implies that the while loop starting on line 12 produces
    infinitely many rewriting rules $(P_i,Q_i), i = 1,2,\dots$. Write
    $\mathcal{U}$ for the set of these rules.

    \textbf{Claim.} $\mathcal{U}$ is confluent and generates $\sim$.

    \emph{Proof of the Claim.} $\mathcal{U}$ generates $\sim$ by the same proof
    that shows $\mathcal{S}$ does. If $\mathcal{U}$ is not confluent, put \[ W
    := \min\{V : \text{local confluence fails at } V\}.\] Proposition
    \ref{localfailure} guarantees us that $W = ABC$ with $B\neq \varepsilon$
    and one of the following is true:

    (1) $W = P_j, B = P_i, j < i$ and no word can be derived from both $AQ_iC$
    and $Q_j$ by means of $\mathcal{U}$.

    (2) $AB = P_i, BC = P_j$ and no word can be derived from both $AQ_j$ and
    $Q_iC$ be means of $\mathcal{U}$.

    If (1) holds then during the execution of \textsc{KnuthBendix}, a call to
    \textsc{Overlap}$(j,i)$ is made. Once this call has returned, there is a
    word $W'$ which may be derived from both $AQ_iC$ and $Q_j$ using
    $\mathcal{S}$ in its current state. Since $\mathcal{S} \subset
    \mathcal{U}$, the word $W'$ is derivable by means of $\mathcal{U}$,
    contrary to (1).

    In case (2) holds, the call to \textsc{Overlap}$(i,j)$ would give rise to a
    word which may be derived from $AQ_j$ and $Q_iC$ by means of $\mathcal{U}$.

    In all cases we have a contradiction and therefore $\mathcal{U}$ is
    confluent. ///

    To complete the proof of the theorem, let $\mathcal{C}$ be the set of
    canonical forms with respect to $\sim$ and put \[\mathcal{P} := \{P \in X^*
    - \mathcal{C} : \text{every proper subword of } P \text{ is in }
    \mathcal{C}\}.\] (This is precisely the same set defined on line 21.)

    If we rewrite $P\in \mathcal{P}$ by means of $\mathcal{U}$, we obtain
    $\overline{P}\in \mathcal{C}$ such that $P \sim \overline{P}$. Therefore
    there is a rule in $\mathcal{U}$ with left side $P$ and this rule is unique
    by the construction of $\mathcal{U}$. Put \[\mathcal{V} := \{(P,Q)\in
    \mathcal{U} : P \in \mathcal{P}\}.\] Since we are given that
    $\mathrm{RC}(X,<,\mathcal{R})$ is finite, it follows that $\mathcal{P}$ and
    $\mathcal{V}$ are finite. To prove this last statement, we consider the
    sequence of rewriting rules starting at $P$ and terminating at an element
    of $\mathrm{RC}(X,<,\mathcal{R})$. Each such chain is finite in length, and
    there are only finitely many such chains. The set $\mathcal{P}$ is the set
    of first elements of such chains, and the set $\mathcal{V}$ is the set of
    first steps in such chains.

    Let $n = \max\{n : (P_n,Q_n) \in \mathcal{V}\}.$ We have $i > n \Rightarrow
    P_i \notin \mathcal{C}$ and is irreducible with respect to $\{(P_j,Q_j) : 1
    \le j < i\}.$ This is a contradiction since $\mathcal{C}$ is the set of
    irreducible elements. Therefore $\mathcal{U}$ is finite and that the while
    loop starting on line 12 terminates.

    The characterization of $\mathrm{RC}(X,<,\mathcal{R})$ in Proposition
    \ref{confluencecharacterization} gives us that the set $\mathcal{T}$
    defined by the for loop starting on line 23 coincides with
    $\mathrm{RC}(X,<,\mathcal{R})$.
\end{proof}
\end{ap}

