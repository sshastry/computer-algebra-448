\documentclass[10pt,a4paper,reqno]{amsart}

\usepackage[T1]{fontenc}
\usepackage{macros}
\usepackage{fullpage}

\usepackage{droid}
\usepackage{eulervm}

\begin{document}
\bibliographystyle{alpha}

\noindent Math 448 Fall 2011, Computer Algebra\\
\noindent Instructor: Sreekar M. Shastry\\
\noindent Solutions to the Mid Semester Examination\\
\noindent 12-Oct-2011 1400-1530 in Room C304 of HR4

\bigskip

\begin{itemize}
    \item There are 6 problems. Each problem is worth 5 points. The maximum score is 30 points.
    \item Clearly state the results you invoke.
\end{itemize}

\bigskip

\noindent 1. Let $M = (S,A,\mu,s_0, Y)$ be a deterministic finite state
automaton over the alphabet $A$, where $S$ is the set of states, $Y$ is the set
of accept states, and $s_0$ is the start state.  Recall the definition of
$\widehat{\mu}$ given inductively as follows: $\widehat{\mu}(q, \varepsilon) :=
q$ and $\widehat{\mu}(q, ua) := \mu(\widehat{\mu}(q,u),a)$ for $u\in A^*, a\in
A$.

Suppose that there exists an $a\in A$ such that for all $q\in S$ we have
$\mu(q,a) = q$.

(a) Show that $\widehat{\mu}(q,a^n) = q$ for all $n \ge 0$ where $a^n$ is the
string consisting of $n$ $a$'s.

(b) Show that either $\{a\}^* \subseteq L(A)$ or $\{a\}^* \cap L(A) =
\varnothing.$

\bigskip

\emph{Solution.} (a) Proof by induction. We have been given the base case
$\widehat{\mu}(q,a) = \mu(q,a) = q$. Assume the statement is true for $n-1$.
Then $\widehat{\mu}(q^n,a) = \mu(\widehat{\mu}(q,a^{n-1}),a) = \mu(q,a) = q$,
as required.

(b) If $\{a\}^*\cap L(M) \neq \varnothing$ then $a^k\in L(M)$ for some $k$ and
$\mu(s_0, a^k)$ is an accept state. This implies that the start state is an
accept state since $\widehat{\mu}(s_0, a) = s_0$ by assumption. Thus for any $j
\ge 0$ we have $\mu(s_0,a^j) = s_0$ is an accept state and therefore \(\{a\}^*
\subset L(M)\), as required.

\bigskip

\noindent 2. We are given a deterministic automaton $M=(S,A,\mu,s_0, Y).$

(a) Show that $$\widehat{\mu}(q,xy) = \widehat{\mu}(\widehat{\mu}(q,x),y)$$ for
any state $q$ and strings $x,y\in A^*$. Hint: use induction on $|y|$.

(b) Show that for any $q\in S, x\in A^*, a\in A$ we have \[\widehat{\mu}(q,ax)
= \widehat{\mu}(\mu(q,a),x).\] Hint: use part (a).

\bigskip

\emph{Solution.} (a) We use induction on $|y|$. The base case $|y|=1$ is just
the definition of $\widehat{\mu}$. Assume the statement is true for a given
$y_0\in A^*.$ Then we must prove it for $y = y_0a$ for any $a\in A$. (One could
intepret this as a proof by structural induction over $y\in A^*$ rather than
induction on $|y|$.)

We have \begin{align*}
\widehat{\mu}(q,xy) &= \widehat{\mu}(q,xy_0a)\\
&= \mu(\widehat{\mu}(q,xy_0),a) & \text{definition}\\
&= \mu(\widehat{\mu}(\widehat{\mu}(q,x),y_0),a) & \text{induction hypothesis} \\
&= \widehat{\mu}(\widehat{\mu}(q,x),y_0a) & \text{induction hypothesis}\\
&= \widehat{\mu}(\widehat{\mu}(q,x),y),
\end{align*}
as required. Note that we had to use the induction hypothesis twice.

(b) We have
\begin{align*}
    \widehat{\mu}(q,ax) &= \widehat{\mu}(\widehat{\mu}(q,a),x) & \text{by (a)}\\
    &= \widehat{\mu}(\mu(q,a),x) & \text{by definition},
\end{align*} as required.

\bigskip

\noindent 3. Write a regular expression for the following languages.

(a) The set of strings over the alphabet \(\{a,b,c\}\) containing at least one
$a$ and at least one $b$.

(b) The set of strings of 0's and 1's whose third symbol from the right end is
a 1.

\bigskip

\emph{Solution.} (a) Let $r$ be the regular expression $(\varepsilon+a+b+c)$.
Then the regular expression we are looking for is $$r^* a r^* b r^* + r^* b r^*
a r^*.$$

(b) The sought regular expression is \[(\varepsilon+0+1)^*1(0+1)(0+1).\]
\bigskip

\bigskip

\noindent 4. Let $\mathscr{L}$ be the set of strings of balanced parentheses.
Thus $\mathscr{L}$ consists of the strings of characters ``('' and ``)'' that
can appear in a well-formed arithmetic expression. Show that $\mathscr{L}$ is
not a regular language.

\bigskip

\emph{Solution.} We use the pumping lemma. Suppose for contradiction that
$\mathscr{L}$ is regular. Let $n$ be such that given $w \in \mathscr{L}$ of
length $\ge n$ there exists $x,y,z$ with $y \neq \varepsilon$ and $|xy| \le n$
such that  $w = xyz$ and $xy^iz \in L$ for all $i$.

Define \[w_j := \underbrace{( ( \cdots (}_{j} \underbrace{) \cdots ) )}_{j}\]
and consider $w_N$ for some $N > n$. Then there exists $x,y,z$ as in the
pumping lemma. But since $|xy| \le n < N$ we must have that $y$ consists
entirely of left parentheses. Thus for $i \ge 2$, $xy^iz$ will have more left
parentheses than right parentheses and therefore cannot be a balanced string.
This contradiction shows that $\mathscr{L}$ is not a regular language.
\bigskip

\noindent 5. Let $G$ be a group.

(a) If $N \triangleleft G$ is a normal subgroup, show that \[\{(ng,g) : n\in N,
g\in G\}\] is a subgroup of $G\times G$.

(b) Show that the construction in part (a) establishes a bijection between the
set of normal subgroups of $G$ and the set of subgroups of $G\times G$ which
contain the diagonal subgroup $\Delta := \{(g,g)\in G\times G : g\in G\}$.

\bigskip

\emph{Solution.} (a) Let $H_N := \{(ng,g) : n\in N, g\in G\}$. To be a subgroup
of $G\times G$ we must check that $H_N$ has identity, multiplication, and
inverse coming from $G\times G$. To see that $(1,1) \in H_N$ just take $n=1 \in
N \subset G, g = 1\in G$. To see that $H_N$ is closed under multiplication we
compute for $u,v\in N, g,h\in G$ \[(ug,g)(vh,h) = (ugvh,gh) = (u(gvg^{-1})gh,
gh) \in H_N\] since $N$ is normal in $G$ so that $gvg^{-1} \in N$. To see that
$H_N$ is closed under inversion we compute \[(ng,g)^{-1} = ( (ng) ^{-1}, g^{-1}
) = (g^{-1} n^{-1}, g^{-1}) = ( (g^{-1} n^{-1} g) g^{-1} , g^{-1}) \in H_N\] as
before.

(b) Let $S := \{\text{normal subgroups } N \triangleleft G\}$ and $T :=
\{\text{subgroups } H \subset G\times G \text{ s.t. } H \supset \Delta\}$. We
define a map $\alpha : S \rightarrow T$ by $\alpha(N) = H_N$ where $H_N :=
\{(ng,g) : n\in N, g\in G\}$ as in part (a), and we define $\beta : T
\rightarrow S$ by $\beta(H) := N_H := \{ab^{-1} : (a,b) \in H\}$.

We must show that $\alpha \circ \beta = \mathrm{id}_T$ and $\beta \circ \alpha
= \mathrm{id}_S$. This is the definition of a bijective correspondence. We
compute \[\beta(\alpha(N)) = \beta(H_N) = \beta( \{(ng,g) : n\in N, g\in G\}  )
= \{n g g^{-1} : n \in N, g \in G\} = N\] and therefore $\beta\circ \alpha =
\mathrm{id}_S$. Next, we have \( \alpha(\beta(H)) = \alpha(N_H) = \{
(ab^{-1}g,g) : (a,b) \in H, g\in G \}.\) Therefore we must show that under
the assumption $H \supset \Delta$ we have \[\{ (ab^{-1}g,g) : (a,b) \in H,
g\in G \} = \{(a,b) \in H\}.\] To prove $\supset$ it suffices to take $g=b$
in the left hand side. To prove $\subset$ we use the fact that $\Delta
\subset H$. We then have $$(ab^{-1}g,g) = (ab^{-1},1)(g,g) =
(a,b)(b^{-1},b^{-1})(g,g) \in H\cdot \Delta \cdot \Delta \subset H$$ as
required.

\bigskip

\noindent 6. Construct a deterministic automaton which accepts the following
language over \(\{0,1\}\): \[\mathscr{L} := \{x \in \{0,1\}^* : x \text{
represents a multiple of 3 in binary}\}.\] Leading zeros are permitted, and
$\varepsilon$ represents the number 0. For example, the string 001001
represents the number 0+0+8+0+0+1=9 and thus $001001 \in \mathscr{L}$.

\bigskip

\emph{Solution.} To determine whether a number is a multiple of 3, we must
compute the residue class of the number mod 3. Thus given a string we must
decide which residue class mod 3 it lands in and accept the string iff it lands
in the residue class of 0 mod 3. Therefore our set of states is the set of
congruence classes mod 3: $S := \{q(0), q(1), q(2)\}$ where $q(i)$ indicates
the class of $i \imod{3}$. The start state is $q(0)$ since the empty string
corresponds to the number 0 and therefore the class of 0 mod 3. The set of
accept states is \(\{q(0)\}\).  We claim that the transition matrix of the
sought automaton is
\[
\begin{array}{c|c c}
    & 0 & 1 \\
    \hline
    q(0) & q(0) & q(1)\\
    q(1) & q(2) & q(0) \\
    q(2) & q(1) & q(2).
\end{array}
\] We prove this claim by induction on the length of input string. Define
$\mathrm{eval} : \{0,1\}^* \rightarrow \mathbb{Z}$ by \[\mathrm{eval}(a_n
a_{n-1}\cdots a_1 a_0) := \sum_{i = 0}^n a_i 2^i\] and define
$\mathrm{eval}(\varepsilon) := 0$. Thus $\mathrm{eval}(100) = 4 + 0 + 0 = 4,
\mathrm{eval}(0010) = 0 + 0 + 2 + 0 = 2$, etc. Then our claim is equivalent to
the claim that \[\widehat{\mu}(q(0),w) = q(\mathrm{eval}(w)\imod{3})
\tag{$*$}.\] In other words, the left hand side of $(*)$ is defined by means of
the above transition matrix, and we must show that this agrees with the right
hand side of $(*)$ for each $w\in \{0,1\}^*$.

We prove $(*)$ by induction on the length of $w$. The base case $|w| = 1$
follows from inspecting the above table and comparing with the definition of
$\mathrm{eval}$. Assume the induction hypothesis that $(*)$ is true for $w$. We
must prove it is true for the string $wa$ where $a \in \{0,1\}$. (As so often
happens, one could say that we are doing a structural induction over all
strings instead of a classical induction on the length of the strings; they
amount to the same thing in this proof.)

We have \[\mathrm{eval}(wa) = 2(\mathrm{eval}(w)) + a \tag{$**$}\] where on the
left hand side we understand the $a$ to mean an element 0 or 1 of the alphabet,
and on the right hand side we understand $a$ to be the number 0 or 1 $\in \Z$.
Thus we have
\begin{align*}
\widehat{\mu}(q(0),wa) & = \mu(\widehat{\mu}(q(0),w),a) \\
& = \mu(q(\mathrm{eval}(w)\imod{3}),a) & \text{by the induction hypothesis}\\
& = q(2(\mathrm{eval}(w)) + a \imod{3})& \text{by the definition of } q\\
& = q(\mathrm{eval}(wa) \imod{3}) & \text{by } (**),
\end{align*}
as required.

\end{document}
